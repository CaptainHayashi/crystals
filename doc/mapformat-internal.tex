\documentclass [12pt,a4paper]{article}
\pagestyle{headings}

\usepackage [utf8x]{inputenc}
\usepackage{amsmath}
\usepackage{amsfonts}
\usepackage{amssymb}
\usepackage{amsthm}
\usepackage [margin=0.9in]{geometry}
\usepackage{ucs}
\usepackage{listings}
\usepackage{color}
\usepackage{textcomp}

\lstset{
  language=C,
  keywordstyle=\bfseries\ttfamily\color[rgb]{0,0,1},
  identifierstyle=\ttfamily,
  commentstyle=\color[rgb]{0.133,0.545,0.133},
  stringstyle=\ttfamily\color[rgb]{0.627,0.126,0.941},
  showstringspaces=false,
  basicstyle=\small,
  numberstyle=\footnotesize,
  numbers=left,
  stepnumber=1,
  numbersep=10pt,
  tabsize=2,
  breaklines=true,
  prebreak = \raisebox{0ex}[0ex][0ex]{\ensuremath{\hookleftarrow}},
  breakatwhitespace=false,
  aboveskip={1.5\baselineskip},
  columns=fixed,
  upquote=true,
  extendedchars=true,
}

\newcommand{\superscript} [1]{\ensuremath{^{\textrm{#1}}}}
\newcommand{\subscript} [1]{\ensuremath{_{\textrm{#1}}}}

\author{Matt Windsor}
\title{Internal Map Format Proposal}
\date{}

\begin{document}

\maketitle{}

\begin{abstract}
  This \LaTeX file details a proposal for an internal map format, with
  a test renderer available in the \emph{maprender} branch.

  \emph{Note:} This is just a proposal, and is subject to heavy
  change.

\end{abstract}

\section{The map structure}

\subsection{C structure listing}

Reproduced here is the internal map structure, at the time of writing.

\begin{lstlisting}
typedef unsigned short layer_t; /**< Type for layer data. */

struct map
{
  unsigned int width;       /**< Width of the map, in tiles. */
  unsigned int height;      /**< Height of the map, in tiles. */
  unsigned char num_layers; /**< Number of arrays to store in the map. */
  layer_t **data_layers;    /**< Pointers to map layer arrays. */
};
\end{lstlisting}

\subsubsection{Forthcoming features}

\begin{itemize}
\item Objects (per-map, per-layer or per-tag? Linked list or
  otherwise).
\end{itemize}

\subsubsection{A note on types}

The data layer type is typedef'd as layer\_t to allow quick changes in
the future, should a global limit on 65,536 tiles become too small, or
if layer\_t needs to be changed to a compound structure later.

\subsection{Layers}

The map tiles are organised into \emph{layers}.  The theoretical upper
bound on the number of layers that can be stored is 256 (the maximum
value of an unsigned char).

Layers consist of $(width * height) + 1$ unsigned chars, arranged in a
dynamically allocated array of $num\_layers$ members at
\textbf{num\_layers}.  Each char except the last in each layer
represents one tile, and should be read left-to-right, top-to-bottom,
bottom layer to top layer.

\subsection{Tags}

The last char in each layer is reserved as the \emph{tag byte}.  This
is a number, 0 to 255, that is used to identify the layer to
objects.

Objects also have tag bytes (either by default or specified during
their instantiation on the map) and, when an object is instantiated on
a map, it will look for the first layer with a matching tag byte or,
if the tag byte cannot be found, the first layer defined (the bottom
layer).

The layer found is then treated as the object's ``floor'', with all
layers below the floor ignored and rendered beneath the object, and
all layers above the floor ignored and rendered above the object.
Barring exceptional circumstances, an object can only interact with,
and be blocked by, objects on its ``floor'' layer.

An example of the usage of this would be a scene in which there is a
balcony floor (let us say tag 2) looking out on a floor below (let us
say tag 1).  The player is tag 1 (see ``Defined Tags'' below), and
positioned on the below floor, and another NPC is tag 2 and positioned
on the floor above.  If the player moves underneath the floor, he will
disappear underneath the floor as the balcony is a layer above the
floor, and as the player is tagged with the lower floor's tag, the
player will treat the lower floor as ground and the balcony as
overlay.  At the same time, the NPC will be rendered on top of the
balcony as its tag matches that of the balcony.

\subsubsection{Defined Tags}

\begin{tabular}{ l p{5in} }
  ID & Purpose \\
  \hline
  0  & Inactive layer.  No objects can be instantiated on a tag-0
       layer, and it will not be considered for object checking during 
       rendering. \\
  1  & Recommended default tag for the player and objects that will 
       interact with the player. \\
\end{tabular}

\subsubsection{Changing Tag}

A possible feature would be the ability to change the tag of an object
temporarily on a map.  For example, if the player walks up a ladder,
their tag could change so they consider a different layer as their
floor.

This, of course, would need some added complexities (see
\emph{Arguments against tags} below).

\subsubsection{Rationale for tags}

Tags would allow objects to carry information on where they should, by
default, be placed in a scene, without hardcoding the map layer number
(which could change from map to map.)  For example, the player object
would by default start each map with the tag 1, which could correspond
to layer 0 on one map, layer 2 on another map, or even layer 63 on
another map.

While a particularly complex map may make use of many layers to
achieve graphical effects, it will likely only have a few ``physical''
layers (eg ground, mezzanines, overlay).  Tags would allow for objects
to be placed based on physical layers rather than graphical layers.

If layers are added to or removed from a map, the tags will likely
remain the same while the layer IDs change.

In addition, layers tagged 0 could be completely disregarded while
rendering objects which (dependent on how the rendering system is
implemented) could save effort.

Tags could be used to place animating objects, for example flickering
street lamps, above the player.

\subsubsection{Arguments against tags}

Though I believe that tags are a good idea, for completeness's sake
and the purposes of debate some potential arguments against the
aforementioned tag implementation are discussed.

Tags are a complexity that will be thrust upon the map designer, and
such could be an annoyance especially if all maps have the same layer
layout anyway.

Tags require an extra byte per layer for storage, which is a modest
cost but a cost nonetheless.

The numeric format of tags means that they are less intuitive than,
say, a string-based system (the player treats the first layer named
``player-floor'' as its floor, for example) - but this would add
complexity and space consumption.

If tag changing is made possible, this could become messy as objects
move from their default tag to other tags.  Doors would have to be
scripted to ensure that the player has the right tag when moving to a
new map, and this would break the idea of tags allowing objects to
ignore differences between maps.

Only one tag can be stored per layer under the proposed system.  A
system of being able to chain tags could be implemented, but would add
more complexity to the map structures.

\section {The tileset}

The global tileset image, at time of writing, is located in
\emph{tiles.png}.  Currently, all the layer data provides is an offset
into this tileset image.

Later, there will likely be metadata associated with each tile, such
as collision information.

\end{document}