\documentclass[12pt,a4paper]{article}
\pagestyle{headings}

\usepackage[utf8x]{inputenc}
\usepackage{amsmath}
\usepackage{amsfonts}
\usepackage{amssymb}
\usepackage{amsthm}
\usepackage[margin=0.9in]{geometry}
\usepackage{ucs}
\usepackage{listings}
\usepackage{color}
\usepackage{textcomp}

\lstset{
  language=C,
  keywordstyle=\bfseries\ttfamily\color[rgb]{0,0,1},
  identifierstyle=\ttfamily,
  commentstyle=\color[rgb]{0.133,0.545,0.133},
  stringstyle=\ttfamily\color[rgb]{0.627,0.126,0.941},
  showstringspaces=false,
  basicstyle=\small,
  numberstyle=\footnotesize,
  numbers=left,
  stepnumber=1,
  numbersep=10pt,
  tabsize=2,
  breaklines=true,
  prebreak = \raisebox{0ex}[0ex][0ex]{\ensuremath{\hookleftarrow}},
  breakatwhitespace=false,
  aboveskip={1.5\baselineskip},
  columns=fixed,
  upquote=true,
  extendedchars=true,
}

\newcommand{\superscript}[1]{\ensuremath{^{\textrm{#1}}}}
\newcommand{\subscript}[1]{\ensuremath{_{\textrm{#1}}}}

\author{Michael Walker}
\title{Crystals Option Parser}
\date{}

\begin{document}

\maketitle{}

\begin{abstract}
This \LaTeX file details the usage of the Crystals Option Parser, originally written by Michael Walker, for developers of the crystals engine.

Code examples will be presented.
\end{abstract}

\section{Functions}

The option parser presents one function which should be called by external functions, and also a range of internal functions which should only be called by the option parser.

Any functions which have the \textbf{int} return type return the defined constants \textbf{SUCCESS} or \textbf{FAILURE}.

\subsection{External Functions}

To parse options, the unsurprisingly-named \textbf{parse\_options} function is used. See the sections named \textbf{Options} and \textbf{Usage} for more details.

\begin{lstlisting}
int
parse_options (int argc, const char* argv[], option options[]);
\end{lstlisting}

\subsection{Internal Functions}

To check whether an option is the null end-of-list marker:

\begin{lstlisting}
int
is_null_option (option check);
\end{lstlisting}

To get an option by either its short name or long name from a list of options:

\begin{lstlisting}
option
get_option (const char shortname, const char* longname, option options[]);
\end{lstlisting}

To generate and display the help text, and then exit:

\begin{lstlisting}
void
print_help_text (option options[]);
\end{lstlisting}

To generate and display the version text, and then exit:

\begin{lstlisting}
void
print_version_text (void);
\end{lstlisting}

\section{Options}

At its simplest, the option parser is simply a magic function which takes a list of strings, a list of options, and spits out values. It does this using pointers.

The list of options which is passed to the parser is a list of \textbf{option} structures.

\begin{lstlisting}
option foo = {longname, shortname, type, pointer, helptext};
\end{lstlisting}

The \textbf{longname} of an option is a string. The \textbf{shortname} of an option is a character. An example of a long and short option follow:

\begin{lstlisting}[language=bash]
foo --bar -c
\end{lstlisting}

However, it would be fairly useless if the option parser could only check for the presence of an option, which is what the \textbf{type} field is for. This is one of a number of constants:

\begin{itemize}
  \item OPTION\_PARAM\_NONE
  \item OPTION\_PARAM\_INT
  \item OPTION\_PARAM\_FLOAT
  \item OPTION\_PARAM\_STRING
\end{itemize}

Finally, the \textbf{pointer} is a pointer to the variable to store the value in, and the \textbf{helptext} is text to display when the \textbf{--help} parameter is passed. Here is a more in-depth example, taken from the option parser testsuite:

\begin{lstlisting}
option options[] = {{NULLC, 'a',   OPTION_PARAM_NONE,   &a,
                        "Presence flag."},
                    {NULLC, 'b',   OPTION_PARAM_INT,    &b,
                        "Integer parameter."},
                    {NULLC, 'c',   OPTION_PARAM_FLOAT,  &c,
                        "Float parameter."},
                    {"dah", 'd',   OPTION_PARAM_STRING, &d,
                        "String parameter."},
                    {NULLC, 'e',   OPTION_PARAM_INT,    &e,
                        "Integer parameter."},
                    {"foo", 'f',   OPTION_PARAM_NONE,   &foo,
                        "Presence flag."},
                    {"bar", NULLC, OPTION_PARAM_INT,    &bar,
                        "Integer parameter."},
                    {"baz", NULLC, OPTION_PARAM_FLOAT,  &baz,
                        "Float parameter."},
                    {"qux", 'q',   OPTION_PARAM_STRING, &qux,
                        "String parameter."},
                    {NULLC, NULLC, NULLC, NULLC, NULLC}};
\end{lstlisting}

Note that \textbf{NULLC} is the null character. The last entry in the list of options must be a null entry, to signify that it is the end to the option parser. The parser allows for flexibility with the short option type, for instance these two lines are equivalent:

\begin{lstlisting}[language=bash]
foo -b 1
foo -b1
\end{lstlisting}

However, for the long options, there is only one syntax permitted:

\begin{lstlisting}[language=bash]
foo --bar=5
\end{lstlisting}

\section{Using}

Once you define a list of options, as well as variables to house the results in, calling the option parser is incredibly simple:

\begin{lstlisting}
parse_options (argc, argv, options)
\end{lstlisting}

If there is a problem, \textbf{FAILURE} will be returned, and an error message will be displayed.

\section{Test Suite}

The option parser comes with a test suite, and can be run with the following command:

\begin{lstlisting}[language=bash]
make tests
\end{lstlisting}

If the test fails, a notice will be displayed. In that case, you can debug further with GDB or Valgrind:

\begin{lstlisting}[language=bash]
gdb -ex run -ex backtrace --args ./tests/optionparser
valgrind --tool=memcheck --leak-check=full --show-reachable=yes ./tests/optionparser
\end{lstlisting}

\end{document}