\documentclass[12pt,a4paper]{article}
\pagestyle{headings}

\usepackage[utf8x]{inputenc}
\usepackage{amsmath}
\usepackage{amsfonts}
\usepackage{amssymb}
\usepackage{amsthm}
\usepackage[margin=0.9in]{geometry}
\usepackage{ucs}
\usepackage{listings}
\usepackage{color}
\usepackage{textcomp}

\lstset{
  language=C,
  keywordstyle=\bfseries\ttfamily\color[rgb]{0,0,1},
  identifierstyle=\ttfamily,
  commentstyle=\color[rgb]{0.133,0.545,0.133},
  stringstyle=\ttfamily\color[rgb]{0.627,0.126,0.941},
  showstringspaces=false,
  basicstyle=\small,
  numberstyle=\footnotesize,
  numbers=left,
  stepnumber=1,
  numbersep=10pt,
  tabsize=2,
  breaklines=true,
  prebreak = \raisebox{0ex}[0ex][0ex]{\ensuremath{\hookleftarrow}},
  breakatwhitespace=false,
  aboveskip={1.5\baselineskip},
  columns=fixed,
  upquote=true,
  extendedchars=true,
}

\newcommand{\superscript}[1]{\ensuremath{^{\textrm{#1}}}}
\newcommand{\subscript}[1]{\ensuremath{_{\textrm{#1}}}}

\author{Michael Walker}
\title{Crystals Dynamic Module Loader}
\date{}

\begin{document}

\maketitle{}

\begin{abstract}
This \LaTeX file details the usage of the Crystals Dynamic Module Loader, originally written by Michael Walker, for both developers of modules, and developers of the Crystals engine.

Code examples will be presented as well as an explanation of how to add new classes of modules to the system.
\end{abstract}

\section{Functions}

The module loader contains a set of functions for external use, which are supplemented by internal-use functions. These should not be called upon by any other aspect of the engine, as their behaviour may change unpredictably and without warning as changes are made. However, the behaviour of the external-use functions will remain consistent, or at least compatible, to avoid anything breaking due to changes.

Any functions which have the \textbf{int} return type return the defined constants \textbf{SUCCESS} or \textbf{FAILURE}.

\subsection{External-Use Functions}

To initialise the global structure, \textbf{g\_modules} (more on that later) with function pointers and suchlike, the \textbf{init\_modules} function must be called, specifying the path to the modules.

\begin{lstlisting}
int
init_modules (const char *path);
\end{lstlisting}

Before termination of the program, the \textbf{close\_modules} function should be called to close any open modules and run any module termination code.

\begin{lstlisting}
void
cleanup_modules (void);
\end{lstlisting}

Finally, there are the functions to load the modules:

\begin{lstlisting}
int
load_module_gfx (char* name);
\end{lstlisting}

\subsection{Internal-Use Functions}

Before a module can be loaded, it should be initialised by setting some pointers to \textbf{NULL}. If they are not, undefined behaviour may arise in some situations. This function is used internally by \textbf{init\_modules}.

\begin{lstlisting}
void
module_bare_init (module_data *module);
\end{lstlisting}

To find the path of a module and store it in a variable \textbf{out}, the \textbf{get\_module\_path} function is used.

\begin{lstlisting}
void
get_module_path (const char* module,
                 char** out);
\end{lstlisting}

To open a module and run any initialisation code, there is a choice of \textbf{get\_module\_by\_name} and \textbf{get\_module}, which differ slightly.

\begin{lstlisting}
int
get_module_by_name (const char* name,
                    module_data *module);
int
get_module (const char* modulepath,
            module_data *module);
\end{lstlisting}

Once a module has been opened, function pointers must be generated and stored in the \textbf{g\_modules} struct so that the provided functions can be used. These are found with the \textbf{get\_module\_function} function.

\begin{lstlisting}
int
get_module_function (module_data metadata,
                     const char *function,
                     void **func);
\end{lstlisting}

Before closing a module, any termination code should be run. This function, \textbf{close\_module}, is used by \textbf{close\_modules}.

\begin{lstlisting}
void
close_module (module_data *module);
\end{lstlisting}

\section{Modules Struct, g\_modules}

The global modules struct, \textbf{g\_modules}, contains function pointers and metadata for every module. It stores generics rather than specifics, and so can be used for many different modules which provide the same functionality. Let us consider a simple example possible \textbf{g\_modules}:

\begin{lstlisting}
struct
{
  char* path;

  struct
  {
    module_data metadata;
    void (*draw_tile)(char* imagepath, int x, int y);
    void (*draw_text)(char* text);
    void (*blank_screen)(void);
  } graphics;

  struct
  {
    module_data metadata;
    void (*play)(char* soundfile);
    void (*loop)(char* soundfile, int times);
  } sound;

  struct
  {
    module_data metadata;
    void (*get_key_press)(int *key);
  } input;
} g_modules;
\end{lstlisting}

Here we have a plausible \textbf{g\_modules} struct, containing modules for graphics, sound, and input. As you can see, these are not, say, opengl-graphics, sdl-sound, and xlib-input; they are generic classes, and as such functions can be called without needing to know what modules are loaded:

\begin{lstlisting}
int key;
(*g_modules.input.get_key_press) (&key);
\end{lstlisting}

The struct is populated by module loader functions, which have to be written for each class. For example:

\begin{lstlisting}
void
load_module_graphics (void)
{
  if (g_modules.graphics.metadata.lib_handle == NULL)
    {
      get_module_by_name ("graphics",
                          &g_modules.graphics.metadata);
      get_module_function (g_modules.graphics.metadata,
                           "draw_tile",
                           (void**) &g_modules.graphics.draw_tile);
      get_module_function (g_modules.graphics.metadata,
                           "draw_text",
                           (void**) &g_modules.graphics.draw_text);
      get_module_function (g_modules.graphics.metadata,
                           "blank_screen",
                           (void**) &g_modules.graphics.blank_screen);
    }
}
\end{lstlisting}

Of course, a more useful implementation would take a parameter containing the name of the module. This would allow users to load any modules, as long as they provided binary-compatible functions. When a module is implemented, the \textbf{init\_modules} and \textbf{close\_modules} need to be updated to handle it, for example:

\begin{lstlisting}
void
init_modules (const char *path)
{
  module_bare_init (&g_modules.input.metadata);
}

void
close_modules (void)
{
  close_module(&g_modules.input.metadata);
}
\end{lstlisting}

\section{Modules}

Modules used in Crystals are .so files, and so need to be compiled in a certain way.

\begin{lstlisting}[language=bash]
clang -Wall -fPIC -rdynamic -c module.c -o module.o
clang -shared -Wl,-soname,module.so -o module.so module.o
\end{lstlisting}

Modules consist of a collection of functions (see the test module in tests/modules/ for an example) which are used by the main engine, via the module loader. Modules can, optionally, contain an \textbf{init} and \ \textbf{term} function, which are run when the module is loaded and unloaded, respectively. These should have the following prototypes:

\begin{lstlisting}
void
init (void);
void
term (void);
\end{lstlisting}

The return type is not essential, but any return values will be ignored, and no parameters will be passed. If you require a more flexible solution, you will have to define your own functions (perhaps named \textbf{start} and \textbf{stop}) and call them manually via the \textbf{g\_modules} struct.

\section{Test Suite}

The module loader comes with a test suite, and can be run with the following command:

\begin{lstlisting}[language=bash]
make tests
\end{lstlisting}

If the test fails, a notice will be displayed. In that case, you can debug further with GDB or Valgrind:

\begin{lstlisting}[language=bash]
gdb -ex run -ex backtrace --args ./tests/module
valgrind --tool=memcheck --leak-check=full --show-reachable=yes ./tests/module
\end{lstlisting}

\end{document}