\documentclass [12pt,a4paper]{article}
\pagestyle{headings}

\usepackage [utf8x]{inputenc}
\usepackage{amsmath}
\usepackage{amsfonts}
\usepackage{amssymb}
\usepackage{amsthm}
\usepackage [margin=0.9in]{geometry}
\usepackage{ucs}
\usepackage{listings}
\usepackage{color}
\usepackage{textcomp}

\lstset{
  language=C,
  keywordstyle=\bfseries\ttfamily\color[rgb]{0,0,1},
  identifierstyle=\ttfamily,
  commentstyle=\color[rgb]{0.133,0.545,0.133},
  stringstyle=\ttfamily\color[rgb]{0.627,0.126,0.941},
  showstringspaces=false,
  basicstyle=\small,
  numberstyle=\footnotesize,
  numbers=left,
  stepnumber=1,
  numbersep=10pt,
  tabsize=2,
  breaklines=true,
  prebreak = \raisebox{0ex}[0ex][0ex]{\ensuremath{\hookleftarrow}},
  breakatwhitespace=false,
  aboveskip={1.5\baselineskip},
  columns=fixed,
  upquote=true,
  extendedchars=true,
}

\newcommand{\superscript} [1]{\ensuremath{^{\textrm{#1}}}}
\newcommand{\subscript} [1]{\ensuremath{_{\textrm{#1}}}}

\author{Matt Windsor}
\title{The Crystals Map Format\\Version 1}
\date{}

\begin{document}

\maketitle{}

\begin{abstract}
  This \LaTeX file details the map format supported at the time of
  writing by Crystals and the Crystals map editor.

  \emph{Note:} This is a proposal, and is subject to heavy
  change.

\end{abstract}

\section{Brief}

\subsection{Requirements}

The map format has been designed to meet the following criteria:

\begin{itemize}

  \item Be easily parsed both by the Crystals engine and by external
    tools, without the need of non-standard libraries (thus making
    format parsing readily cross-platform);

  \item Allow for ample scalability, though not necessarily unlimited
    flexibility;

  \item Support maps of comparable size and standards to those in the
    existing reference games.

\end{itemize}

\subsection{Features}

Version 1 of the map format boasts the following features:

\begin{itemize}

  \item Binary format that can be parsed with C and Java standard
    library functions working at the byte level;

  \item Does not require external libraries to parse;

  \item Support (theoretical) for maps of width and height up to 65,535
    tiles (16 bits per dimension) each;

  \item Support (theoretical) for up to 65,536 layers (16 bits for
    layer count), which are to be composited by the map renderer;

  \item Support for up to 65,535 \emph{layer tags} (16 bits per tag
    field), which can be assigned to layers to provide the map
    renderer with hints as to where to render objects in the layer
    stack without tying objects to layer numbers.

  \item Support for up to 65,536 tiles in one tileset.  At time of
    writing, the tileset path is not part of the map data, but this
    will easily be insertable at the end of the current data
    definition as a null-terminated string.

\end{itemize}

It is believed at the current time that the hard limits, which are
based around the C unsigned short type, are more than sufficient for
Crystals maps.  Further, incompatible, revisions to the map structure
may change this.


\section{The format}

Following is the format specification proper, starting with the
outline of the structure of a map file.  The rest of the specification
is composed in a bottom-up manner, beginning with the representation
of concrete data types and continuing to the definition of the
building blocks of a map.


\subsection {Outline}

\begin{enumerate}

  \item Map header (16 bytes)
  
  \begin{itemize}

    \item ASCII ``CMFT'' identifier (4 bytes)
    \item Format version number (2 bytes)
    \item Width of map in tiles (2 bytes)
    \item Height of map in tiles (2 bytes)
    \item Index of highest layer (number of layers - 1) (2 bytes)
    \item Index of highest zone (number of zones - 1) (2 bytes)
    \item Reserved (2 bytes)

  \end{itemize}


  \item Layer tag data ($4 + (layercount\cdot{}2)$ bytes) 
  
  \begin{itemize}

    \item ASCII ``TAGS'' identifier (4 bytes)
    \item Layer tag (2 bytes per layer)

  \end{itemize}


  \item Layer value data ($4 + (layercount\cdot{}width\cdot{}height\cdot{}2)$ bytes) 
  
  \begin{itemize}

    \item ASCII ``VALS'' identifier (4 bytes)
    \item Layer value data (\emph{height} rows of \emph{width} tile entries,
      each tile constituting 2 bytes; no row end delimitation;
      repeated for each layer in the map with no layer end delimitation)

  \end{itemize}


  \item Layer zone data ($4 + (layercount\cdot{}width\cdot{}height\cdot{}2)$ bytes) 
  
  \begin{itemize}

    \item ASCII ``ZONE'' identifier (4 bytes)
    \item Layer zone data (\emph{height} rows of \emph{width} tile entries,
      each tile constituting 2 bytes; no row end delimitation;
      repeated for each layer in the map with no layer end delimitation)

  \end{itemize}


  \item Zone parameters ($3 + (zonecount\cdot{}2)$ bytes)


  \item End of file - future versions may place additional data here.


\end{enumerate}


\subsection{Unsigned short representation}

Almost all fields are represented as two-byte unsigned short values,
representing values from 0 to 65,535.

These are encoded in a big-endian manner: that is, the highest bit is
stored first, followed by the lowest bit.


\subsection{ASCII identifiers}

Three-character identifiers are placed before the start of data blocks
in order to aid in validation.  In this version of the map editor,
they are not used to detect the nature of blocks, and thus all maps
must include all data blocks in the above order.

The identifiers are encoded in ASCII with one byte per character,
yielding three bytes per identifier.


\subsection{Tile value}

The \emph{value} of a tile denotes the offset into the tile-set at
which the tile image can be found.

It is an unsigned short number and thus each tile in the map
contributes two bytes per layer towards the size of the value layer
plane.


\subsection{Tile zone}

The \emph{zone} of a tile is used to group tiles with similar
properties.  For example, zones can be used to denote impassable
areas, or random encounter areas.

It is an unsigned short number and thus each tile in the map
contributes two bytes per layer towards the size of the value zone
plane.


\subsection{Layer plane}

The tile value and zone attributes are stored in what is referred to
as \emph{layer planes}.

A layer plane is comprised of the tile information for one layer, laid
out in a left-to-right, top-to-bottom format; each row is stored one
by one with no delimiter to mark the end of a row.

Layer planes are generally placed together sequentially without
layer-end delimiters in blocks of the same plane type.

There are two types of layer plane defined in this version of the map
format: the \emph{layer value plane}, which stores \emph{tile
  values}, and the \emph{layer zone plane}. which stores \emph{tile
  zones}.


\section{Implementation details}

\subsection{Java}

A class library for interacting with the Crystals map format is
included in the Java Crystals Map Editor.  This library provides an
object-orientated view of the Crystals map format, in contrast to the
data structure-orientated approach used in the C engine.

\subsection{C structure listing}

Reproduced here is the internal map structure, at the time of writing,
as used in the Crystals engine.  Internally, it is referred to as the
type \emph{map\_t}.


\begin{lstlisting}

typedef unsigned short layer_t; /**< Type for layer data. */
typedef unsigned short tag_t;   /**< Type for layer tags. */

struct map
{
  unsigned short width;       /**< Width of the map, in tiles. */
  unsigned short height;      /**< Height of the map, in tiles. */
  unsigned short num_layers;  /**< Number of arrays to store in the
                                   map. */
  tag_t   layer_tags[];       /**< Array of layer tags. */
  layer_t **data_layers;      /**< Pointers to map layer arrays. */
};

typedef struct map map_t;

\end{lstlisting}

\subsubsection{A note on types}

The data layer type is typedef'd as layer\_t to allow quick changes in
the future, should a global limit on 65,536 tiles become too small, or
if layer\_t needs to be changed to a compound structure later.

The same logic is used with layer tags and tag\_t.

\subsection{Layers}

The map tiles are organised into \emph{layers}.  The theoretical upper
bound on the number of layers that can be stored is 65,536 (the maximum
value of an unsigned short); however, it is expected that most maps
will feature on average 3-5 layers.

Layers consist of $(width * height)$ unsigned shorts, arranged in a
dynamically allocated array of $num\_layers$ members at
\textbf{num\_layers}.  Each value represents one tile, and should be
read left-to-right, top-to-bottom, bottom layer to top layer.

\subsection{Tags}

The last char in each layer is reserved as the \emph{tag byte}.  This
is a number, 0 to 65,535, that is used to identify the layer to
objects.

Objects also have tag indices (either by default or specified during
their instantiation on the map) and, when an object is instantiated on
a map, it will look for the first layer with a matching tag index or,
if the tag index cannot be found, the first layer defined (the bottom
layer).

The layer found is then treated as the object's ``floor'', with all
layers below the floor ignored and rendered beneath the object, and
all layers above the floor ignored and rendered above the object.
Barring exceptional circumstances, an object can only interact with,
and be blocked by, objects on its ``floor'' layer.

An example of the usage of this would be a scene in which there is a
balcony floor (let us say tag 2) looking out on a floor below (let us
say tag 1).  The player is tag 1 (see ``Defined Tags'' below), and
positioned on the below floor, and another NPC is tag 2 and positioned
on the floor above.  If the player moves underneath the floor, he will
disappear underneath the floor as the balcony is a layer above the
floor, and as the player is tagged with the lower floor's tag, the
player will treat the lower floor as ground and the balcony as
overlay.  At the same time, the NPC will be rendered on top of the
balcony as its tag matches that of the balcony.

\subsubsection{Defined Tags}

\begin{tabular}{ l p{5in} }
  ID & Purpose \\
  \hline
  0  & Inactive layer.  No objects can be instantiated on a tag-0
       layer, and it will not be considered for object checking during 
       rendering.  This is often referred to as the \emph{null tag.} \\
  1  & Recommended default tag for the player and objects that will 
       interact with the player. \\
\end{tabular}

\subsubsection{Changing Tag}

A possible feature would be the ability to change the tag of an object
temporarily on a map.  For example, if the player walks up a ladder,
their tag could change so they consider a different layer as their
floor.

This, of course, would need some added complexities (see
\emph{Arguments against tags} below).

\subsubsection{Rationale for tags}

Tags would allow objects to carry information on where they should, by
default, be placed in a scene, without hardcoding the map layer number
(which could change from map to map.)  For example, the player object
would by default start each map with the tag 1, which could correspond
to layer 0 on one map, layer 2 on another map, or even layer 63 on
another map.

While a particularly complex map may make use of many layers to
achieve graphical effects, it will likely only have a few ``physical''
layers (eg ground, mezzanines, overlay).  Tags would allow for objects
to be placed based on physical layers rather than graphical layers.

If layers are added to or removed from a map, the tags will likely
remain the same while the layer IDs change.

In addition, layers tagged 0 could be completely disregarded while
rendering objects which (dependent on how the rendering system is
implemented) could save effort.

Tags could be used to place animating objects, for example flickering
street lamps, above the player.

\subsubsection{Arguments against tags}

Though it is the author's belief that tags are a good feature, for
completeness's sake and the purposes of debate some potential
arguments against the aforementioned tag implementation are discussed.

Tags are a complexity that will be thrust upon the map designer, and
such could be an annoyance especially if all maps have the same layer
layout anyway.

Tags require an extra byte per layer for storage, which is a modest
cost but a cost nonetheless.

The numeric format of tags means that they are less intuitive than,
say, a string-based system (the player treats the first layer named
``player-floor'' as its floor, for example) - but this would add
complexity and space consumption.

If tag changing is made possible, this could become messy as objects
move from their default tag to other tags.  Doors would have to be
scripted to ensure that the player has the right tag when moving to a
new map, and this would break the idea of tags allowing objects to
ignore differences between maps.

Only one tag can be stored per layer under the proposed system.  A
system of being able to chain tags could be implemented, but would add
more complexity to the map structures.

\section {The tileset}

The global tileset image, at time of writing, is located in
\emph{tiles.png}.  Currently, all the layer data provides is an offset
into this tileset image.

\end{document}